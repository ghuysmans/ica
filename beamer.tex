\documentclass[compress]{beamer}
\usepackage{graphicx}
\usepackage{enumerate}
\usepackage{amssymb}
\usepackage{cancel}
\usepackage[utf8]{inputenc}
\usepackage{color}
\usepackage{dcolumn}
\newcolumntype{d}[1]{D{.}{.}{#1}}
\usepackage[francais]{babel}
\newcommand{\esN}{\mathbb{N}}
\newcommand{\esR}{\mathbb{R}}
\usetheme[navigation]{UMONS}
\title{Analyse en composantes indépendantes}
\date{25 mai 2016}
\author[G. \textsc{Huysmans}, M. \textsc{Lempereur}]
	{Guillaume \textsc{Huysmans}, Martin \textsc{Lempereur}}
\institute[]{Département d'Informatique \\
	Université de Mons \\[2ex]\includegraphics[height=4ex]{UMONS}\hspace{2em}
	\raisebox{-1ex}{\includegraphics[height=6ex]{UMONS_FS}}}
\bibliographystyle{plain}
\begin{document}
\begin{frame}
	\maketitle
\end{frame}
\begin{frame}
	\frametitle{Plan}
	\tableofcontents
\end{frame}


\section{Résumé théorique}
\begin{frame}
	\frametitle{Problème de la soirée cocktail}
	Plusieurs micros $x_i$ sont disposés parmi
	différentes personnes $s_j$ qui parlent en même temps.
	L'objectif est de calculer une approximation de $s_j$
	ainsi que la matrice de mixage $A$ afin que $AS=X$.

	\begin{block}{Linéarité}
	Ainsi, nous ferons l'hypothèse que chaque micro capture
	une combinaison linéaire de ce que les gens (sources) émettent :
	\[
		\forall i\in\esN \qquad x_i = \sum_{j=0}^k a_{ij} s_j
	\]
	\end{block}
	\pause

	Si nous disposions de $A$, le problème serait simple :
	\[
		\cancel{(A^{-1}A)}S = A^{-1}X
	\]
\end{frame}

\begin{frame}
	\frametitle{Exemple numérique}
	\[
		\underbrace{\left(\begin{array}{lll}
			1 & 1 & 1 \\
			\color{red}1 & \color{red}3 & \color{red}1 \\
			3 & 1 & 0
		\end{array}\right)}_{\text{A }(3\times3)}
		\underbrace{\left(\begin{array}{ccccc}
			0 & 1 & \color{green}2 & 3 & 4 \\
			5 & 5 & \color{green}0 & 5 & 5 \\
			1 & -1 & \color{green}1 & -1 & 1
		\end{array}\right)}_{\text{S }(3\times5)}=
		\underbrace{\left(\begin{array}{rrrrr}
			6 & 5 & 3 & 7 & 10 \\
			16 & 15 & \color{blue}3 & 17 & 20 \\
			5 & 8 & 6 & 14 & 17
		\end{array}\right)}_{\text{X }(3\times5)}
	\]

	Chaque ligne $\color{red}a_i$ (coefficients pour le micro $i$) est multipliée
	par une~colonne $\color{green}s_j$ (échantillons-sources en $t=j$)
	afin d'obtenir l'{\color{blue}amplitude} mesurée
	par le~micro $i$ en le temps $j$.
	$X$ contient ainsi un vecteur-ligne par micro.
	\pause

	On peut retrouver $S$ à partir de $A$ et de $X$ (c'est équivalent à la
	résolution d'un système d'équations linéaires) :
	\[
		S\approx
		\left(\begin{array}{d{3.1}d{3.1}d{3.1}}
			0.17 & -0.17 & 0.33 \\
			-0.50 & 0.50 & 0 \\
			1.33 & -0.33 & -0.33
		\end{array}\right)
		\left(\begin{array}{rrrrr}
			6 & 5 & 3 & 7 & 10 \\
			16 & 15 & 3 & 17 & 20 \\
			5 & 8 & 6 & 14 & 17
		\end{array}\right)
	\]
\end{frame}

\begin{frame}
	\frametitle{Limites}
	À cause de la structure du problème, une technique de résolution,
	quelle qu'elle soit, ne permettra pas de tout déterminer :
	\pause
	\begin{enumerate}
	\item L'ordre des sources ne peut pas être déterminé.
	\pause
		\\\textit{Intuition : l'addition est commutative et nous n'avons que
			des sommes pondérées à disposition.}
	\pause
	\item L'amplitude absolue des sources ne peut pas être déterminée.
	\pause
		\\Une source atténuée peut être amplifiée au mixage :
			multiplier une~colonne de $A$ par $k\in\esR_0$ et
			diviser une ligne de $S$ par celui-ci ne change rien à
			la valeur de $X$.
	\pause
	\end{enumerate}

	Ces informations ont en fait été perdues lors du mixage.
\end{frame}


\section{Implémentation}
\begin{frame}
	\frametitle{FastICA}
	Le mixage est modélisé dans l'autre sens :
	\[
		S^TA^T=X^T
	\]
	%TODO match the doc
\end{frame}


\section{Exemples}
\begin{frame}
	\frametitle{Sinusoïdes}
	\begin{enumerate}
	\item first
	\item second
	\end{enumerate}
\end{frame}


\begin{frame}
	\frametitle{Extraits audio}
	\begin{enumerate}
	\item first
	\item second
	\end{enumerate}
\end{frame}


\appendix
\begin{frame}
	\frametitle{Références}
	La théorie est un résumé de l'article suivant :
	\bibliography{article}
	\vspace{2em}

	Le module R utilisé est
	\href{https://cran.r-project.org/web/packages/fastICA/}{FastICA},
	développé par J.~L.~Marchini, C.~Heaton et B.~D.~Ripley et
	se base également sur \cite{hyv}.
\end{frame}


\end{document}
