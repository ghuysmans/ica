\documentclass[compress]{beamer}
\usepackage{graphicx}
\usepackage{enumerate}
\usepackage{amssymb}
\usepackage{cancel}
\usepackage[utf8]{inputenc}
\usepackage[T1]{fontenc}
\usepackage{color}
\usepackage{dcolumn}
\newcolumntype{d}[1]{D{.}{.}{#1}}
\usepackage[francais]{babel}
\usepackage{tikz}
\usepackage{soul}
\newcommand{\esN}{\mathbb{N}}
\newcommand{\esR}{\mathbb{R}}
\usetheme[navigation]{UMONS}
\setbeamertemplate{caption}[numbered]
\title{Analyse en composantes indépendantes}
\date{25 mai 2016}
\author[G. \textsc{Huysmans}, M. \textsc{Lempereur}]
	{Guillaume \textsc{Huysmans}, Martin \textsc{Lempereur}}
\institute[]{Département d'Informatique \\
	Université de Mons \\[2ex]\includegraphics[height=4ex]{UMONS}\hspace{2em}
	\raisebox{-1ex}{\includegraphics[height=6ex]{UMONS_FS}}}
\bibliographystyle{plain}
\begin{document}
\begin{frame}
	\maketitle
\end{frame}
\begin{frame}
	\frametitle{Plan}
	\tableofcontents
\end{frame}


\section{Résumé théorique}
\begin{frame}
	\frametitle{Problème de la soirée cocktail}
	Plusieurs micros $x_i$ sont disposés parmi
	différentes personnes $s_j$ qui parlent en même temps.
	L'objectif est de calculer une approximation de $s_j$
	ainsi que la matrice de mixage $A$ afin que $AS=X$.

	\begin{block}{Linéarité}
	Ainsi, nous ferons l'hypothèse que chaque micro capture
	une combinaison linéaire de ce que les gens (sources) émettent :
	\[
		\forall i\in\esN \qquad x_i = \sum_{j=0}^k a_{ij} s_j
	\]
	\end{block}
	\pause

	Si nous disposions de $A$, le problème serait simple :
	\[
		\cancel{(A^{-1}A)}S = A^{-1}X
	\]
\end{frame}

\begin{frame}
	\frametitle{Exemple numérique}
	\[
		\underbrace{\left(\begin{array}{lll}
			1 & 1 & 1 \\
			\color{red}1 & \color{red}3 & \color{red}1 \\
			3 & 1 & 0
		\end{array}\right)}_{\text{A }(3\times3)}
		\underbrace{\left(\begin{array}{ccccc}
			0 & 1 & \color{green}2 & 3 & 4 \\
			5 & 5 & \color{green}0 & 5 & 5 \\
			1 & -1 & \color{green}1 & -1 & 1
		\end{array}\right)}_{\text{S }(3\times5)}=
		\underbrace{\left(\begin{array}{rrrrr}
			6 & 5 & 3 & 7 & 10 \\
			16 & 15 & \color{blue}3 & 17 & 20 \\
			5 & 8 & 6 & 14 & 17
		\end{array}\right)}_{\text{X }(3\times5)}
	\]

	Chaque ligne $\color{red}a_i$ (coefficients pour le micro $i$) est multipliée
	par une~colonne $\color{green}s_j$ (échantillons-sources en $t=j$)
	afin d'obtenir l'{\color{blue}amplitude} mesurée
	par le~micro $i$ en le temps $j$.
	$X$ contient ainsi un vecteur-ligne par micro.
	\pause

	On peut retrouver $S$ à partir de $A$ et de $X$ (c'est équivalent à la
	résolution d'un système d'équations linéaires) :
	\[
		S\approx
		\left(\begin{array}{d{3.1}d{3.1}d{3.1}}
			0.17 & -0.17 & 0.33 \\
			-0.50 & 0.50 & 0 \\
			1.33 & -0.33 & -0.33
		\end{array}\right)
		\left(\begin{array}{rrrrr}
			6 & 5 & 3 & 7 & 10 \\
			16 & 15 & 3 & 17 & 20 \\
			5 & 8 & 6 & 14 & 17
		\end{array}\right)
	\]
\end{frame}

\begin{frame}
	\frametitle{Limites}
	À cause de la structure du problème, une technique de résolution,
	quelle qu'elle soit, ne permettra pas de tout déterminer :
	\pause
	\begin{enumerate}
	\item L'ordre des sources ne peut pas être déterminé.
	\pause
		\\\textit{Intuition : l'addition est commutative et nous n'avons que
			des sommes pondérées à disposition.}
	\pause
	\item L'amplitude absolue des sources ne peut pas être déterminée.
	\pause
		\\Une source atténuée peut être amplifiée au mixage :
			multiplier une~colonne de $A$ par $k\in\esR_0$ et
			diviser une ligne de $S$ par celui-ci ne change rien à
			la valeur de $X$.
	\pause
	\end{enumerate}

	Ces informations ont en fait été perdues lors du mixage.
\end{frame}
\begin{frame}
	\frametitle{Pré-traitement des données}
	L'algorithme de FastICA demande en entrée des données qui sont traitées.
	Deux opérations doivent être exécutées sur les données :
	\pause
	\begin{enumerate}
	\item Centrage
		\pause
			\\La première action à effectuer est la mise à zéro de la moyenne.
			Pour cela, on la soustrait au vecteur.
			\pause
	\item Blanchissement des données.
		\pause
			\\Le blanchissement des données est une
			technique\footnote{Utilisée aussi pour l'ACP}
			qui consiste à décorréller les données en diagonalisant
			leur matrice de covariance.
	
	\end{enumerate}
\end{frame}
\begin{frame}
	\frametitle{Gaussianité}
	Pour pouvoir faire apparaître des structures plus intéressantes,
	on va vouloir projeter les données sur un espace de faible dimension.
	Le but est d'éloigner les nuages de points les uns des autres.
	Pour y arriver, on va essayer de s'éloigner du cas de données gaussiennes.
	On va minimiser ce qu'on appelle la \textit{néguentropie} :
	plus cette valeur est petite, moins la distribution des données
	est gaussienne.
\end{frame}


\section{Implémentation}
\begin{frame}
	\frametitle{FastICA}
	Intuitivement, le théorème de la limite centrale nous montre que pour un
	certain nombre de V.A.R., la somme de celles-ci se répartit de manière
	gaussienne. Nous allons donc ici essayer de minimiser leur caractère
	gaussien afin de nous rapprocher des sources indépendantes recherchées.
	\pause

	L'algorithme FastICA utilise des heuristiques et il est donc normal
	de ne pas observer le même résultat à chaque exécution.
	
\end{frame}


\section{Exemples}
\begin{frame}
	\frametitle{Sinusoïdes}
	Un des cas les plus simples est la superposition de sinusoïdes.
	\vfill

	\begin{minipage}[b]{.35\textwidth}
	\begin{figure}[h]
	\includegraphics[width=\textwidth]{sine.png}
	\caption{Avant}
	\end{figure}
	\end{minipage}
	\hfill
	\begin{minipage}[b]{.35\textwidth}
	\begin{figure}[h]
	\includegraphics[width=\textwidth]{sine_d.png}
	\caption{Après}
	\end{figure}
	\end{minipage}

	\begin{block}{Observation}
	L'ordre n'est pas préservé et lors de cette exécution,
	la sinusoïde à 660~Hz arrive la première dans le résultat.
	\end{block}
\end{frame}


\begin{frame}
	\frametitle{Extraits audio}
	Deux extraits très différents sont joués en même temps.
	On les suppose indépendants et non gaussiens...
	\begin{figure}[h]
	\includegraphics<1>[height=.65\textheight]{blunt_kavinsky_r.png}
	\includegraphics<2>[height=.65\textheight]{blunt_kavinsky_l.png}
	\includegraphics<3>[height=.65\textheight]{blunt_kavinsky_m2.png}
	\caption{
		\only<1>{James Blunt - Postcards}
		\only<2>{Kavinsky - Night Call}
		\only<3>{\st{Mix} Superposition}
	}
	\end{figure}
\end{frame}


\section{Problèmes pratiques}
\begin{frame}
	\frametitle{Problèmes pratiques}
	L'algorithme FastICA ne gère pas la désynchronisation des mixages.
	Si il y a un léger décalage entre les deux mixage, l'algorithme ne
	fonctionnera pas.
	\pause
	
	\begin{block}{Solution}
	Comme dans le cinéma où le son et l'image étaient séparés (du moins
	jusqu'à il y a peu), on réalisait un <<~clap~>> pour synchroniser
	les deux parties.
	\end{block}
	\pause

	Un autre problème qui se pose est celui de l'écho : on enregistre une
	somme de convolutions de nos sources avec la réponse impulsionnelle de
	la pièce alors que le modèle ne prévoit pas ce cas-là.
	\pause
	
	\begin{block}{Solution}
	Enregistrer dans de meilleures conditions ?
	\end{block}
\end{frame}


\appendix
\begin{frame}
	\frametitle{Références}
	La théorie est un résumé de l'article suivant :
	\bibliography{article}
	\vspace{2em}

	Le module R utilisé est
	\href{https://cran.r-project.org/web/packages/fastICA/}{FastICA},
	développé par J.~L.~Marchini, C.~Heaton et B.~D.~Ripley et
	se base également sur \cite{hyv}.
\end{frame}


\end{document}
